\chapter*{Synopsis}


\begin{figure}[ht]
    \centering
    \includegraphics[scale=0.9]{cover/front.pdf}
\end{figure}


In the last twenty years, aircraft surveillance has moved from controller-based interrogation to automatic broadcast. The Automatic Dependent Surveillance-Broadcast (ADS-B) is the most common method for aircraft to report their state information like identity, position, and speed. Like other Mode~S communications, ADS-B makes use of the 1090 megahertz transponder to transmit data. The protocol for ADS-B is open, and low-cost receivers can easily be used to intercept its signals. Many recent air transportation studies have benefited from this open data source. However, the current literature does not offer a systematic exploration of Mode~S and ADS-B data, nor does it offer an in-depth explanation of the decoding process.

This book tackles this missing area in the literature. It offers researchers, engineers, students, and enthusiasts a clear guide to understanding and making use of open ADS-B and Mode~S data. The first part of this book presents the knowledge required to get started with decoding these signals. It includes background information on primary radar, secondary radar, Mode~A/C, Mode~S, and ADS-B, as well as the hardware and software setups necessary to gather radio signals. After that, the 17 core chapters of the book investigate the details of all types of ADS-B signals and commonly used Mode~S signals. Throughout these chapters, examples and sample Python code are used extensively to explain and demonstrate the decoding process. Finally, the last chapter of the book offers a summary and a brief overview of research topics that go beyond the decoding of these signals.
