\chapter{About the book}

\section{Related resources}\label{related-resources}

This guide document is shared on GitHub and mode-s.org. Please feel free
to help us improving it.

Links to this guide document:

\begin{itemize}
\item
  (Rst source) \url{https://github.com/junzis/the-1090mhz-riddle}
\item
  (Live book) \url{http://mode-s.org}
\end{itemize}

You can download the pyModeS tool from GitHub, which is a Python
implementation of all (and more) message types described here:

\begin{itemize}
\item
  (GitHub) \url{https://github.com/junzis/pyModeS}
\end{itemize}

\section{Contributors}\label{contributors}

From TU Delft:

\begin{itemize}
\item
  Junzi Sun, PhD Candidate, TuDelft
\item
  Huy Vu, Master Student, TuDelft
\item
  Jacco Hoekstra, Prof.dr.ir, TuDelft
\item
  Joost EllerBroek, Dr.ir, TuDelft
\end{itemize}

From GitHub community:

\begin{itemize}
\item
  \url{https://github.com/junzis/the-1090mhz-riddle/graphs/contributors}
\end{itemize}

\section{Contact}\label{contact}

Since the start of the this project, I have received many questions by
email. However, the best way to post your questions is using the
\textbf{GitHub Issues}. This way, your questions and my answers can help
others as well:

\begin{itemize}
\item
  Related with this book:
  \url{https://github.com/junzis/the-1090mhz-riddle/issues}
\item
  Related with pyModeS: \url{https://github.com/junzis/pyModes/issues}
\end{itemize}

Anyhow, still feel free to drop me a messages at:
\textbf{j.sun-1{[}at{]}tudelft.nl}

\section{References}\label{references}

\begin{itemize}
\item
  Technical Provisions for Mode S Services and Extended Squitter.
  International Civil Aviation Organization, 2008.
\item
  Technical Provisions for Mode S Services and Extended Squitter, 2nd
  Edition. International Civil Aviation Organization, 2012.
\item
  Annex 10 to the Convention on International Civil Aviation,
  Aeronautical Telecommunications. International Civil Aviation
  Organization, 2002.
\item
  Minimum Operational Performance Standards for 1090 MHz Extended
  Squitter (DO-260B), RTCA, 2009
\item
  Elementary surveillance (els) and enhanced surveillance (ehs)
  validation via mode s secondary radar surveillance, Project Report
  ATC-337, Lincoln Lab., MIT, 2008.
\item
  Fundamentals of mode s parity coding, tech. rep., Massachusetts
  Institute of Technology, Lincoln Laboratory, 1984.
\item
  \href{https://github.com/antirez/dump1090}{Dump1090 Project}
\item
  \href{http://www.lll.lu/~edward/edward/adsb/VerySimpleADSBreceiver.html}{A
  Very Simple ADSB Receiver}
\end{itemize}
