\section{ADS-B Basics}\label{introduction}

\subsection{Message structure}\label{message-structure}

An ADS-B message is 112 bits long, and consists of 5 parts.

\begin{verbatim}
+--------+--------+-----------+--------------------------+---------+
|  DF 5  |  ** 3  |  ICAO 24  |          DATA 56         |  PI 24  |
+--------+--------+-----------+--------------------------+---------+
\end{verbatim}

Any ADS-B must start with the Downlink Format 17, or 18 in case of TIS-B message. They correspond to 10001 or 10010 in binary for the first 5 bits. Bits 6-8 are used as an additional identifier, which has different meanings within each ADS-B subtype.

In following Table \ref{tb:adsb-structure}, the key information of a ADS-B message is listed.

\begin{table}[!ht]
\centering
\caption{Structure of ADS-B messages}
\label{tb:adsb-structure}
\begin{tabular}{@{}llll@{}}
\toprule
nBits & Bits & Abbr. & Name \\ \midrule
5 & 1 - 5 & DF & Downlink Format \\
3 & 6 - 8 & CA & Capability (additional identifier) \\
24 & 9 - 32 & ICAO & ICAO aircraft address \\
56 & 33 - 88 & DATA & Data \\
 & {[}33 - 37{]} & {[}TC{]} & Type code \\
24 & 89 - 112 & PI & Parity/Interrogator ID \\ \bottomrule
\end{tabular}
\end{table}

It is worth noting that the ADS-B Extended Squitter sent from a Mode S transponder use Downlink Format 17 (\texttt{DF=17}). Non-Transponder-Based ADS-B Transmitting Subsystems and TIS-B Transmitting equipment use Downlink Format 18 (\texttt{DF=18}). By using \texttt{DF=18} instead of \texttt{DF=17}, an ADS-B/TIS-B Receiving Subsystem will know that the message comes from equipment that cannot be interrogated.

An example:

\begin{verbatim}
Raw message in hexadecimal:
8D4840D6202CC371C32CE0576098

[00100]0000010110011
00001101110001110000
110010110011100000

-----+------------+--------------+----------------------+--------------
HEX  | 8D         | 4840D6       | 202CC371C32CE0       | 576098
-----+------------+--------------+----------------------+--------------
BIN  | 10001  101 | 010010000100 | [00100]0000010110011 | 010101110110
     |            | 000011010110 | 00001101110001110000 | 000010011000
     |            |              | 110010110011100000   |
-----+------------+--------------+----------------------+--------------
DEC  |  17    5   |              | [4] ...............  |
-----+------------+--------------+----------------------+--------------
     |  DF    CA  |   ICAO       | [TC] --- DATA -----  | PI
-----+------------+--------------+----------------------+--------------
\end{verbatim}

\subsection{ICAO address}\label{icao-address}

In each ADS-B message, the sender (originating aircraft) can be identified using the ICAO address. It is located from 9 to 32 bits in binary (or 3 to 8 in hexadecimal). In the example above, it is \texttt{4840D6} or \texttt{010010000100}.

An unique ICAO address is assigned to each Mode-S transponder of an aircraft. Thus this is a unique identifier for each aircraft. You can use the query tool (\href{https://junzis.com/adb/}{World Aircraft Database}) from mode-s.org to find out more about the aircraft with a given ICAO address. For instance, using the previous ICAO \texttt{4840D6} example, it will return the result of a \texttt{Fokker\ 70} with registration of \texttt{PH-KZD}.

In addition, you can download the database from the aforementioned website in CSV format.

\subsection{ADS-B message types}\label{ads-b-message-types}

To identify what information is contained in an ADS-B message, we need to take a look at the \texttt{Type\ Code} of the message, indicated at bits 33 - 37 of the ADS-B message (or first 5 bits of the \texttt{DATA} segment).

In following Table \ref{tb:adsb-tc}, the relationships between each \texttt{Type\ Code} and its information contained in the \texttt{DATA} segment are shown.

\begin{table}[!ht]
\centering
\caption{ADS-B Type Code and content}
\label{tb:adsb-tc}
\begin{tabular}{@{}ll@{}}
\toprule
Type Code & Content                              \\ \midrule
1 - 4     & Aircraft identification              \\
5 - 8     & Surface position                     \\
9 - 18    & Airborne position (w/ Baro Altitude) \\
19        & Airborne velocities                  \\
20 - 22   & Airborne position (w/ GNSS Height)   \\
23 - 27   & Reserved                             \\
28        & Aircraft status                      \\
29        & Target state and status information  \\
31        & Aircraft operation status            \\ \bottomrule
\end{tabular}
\end{table}

\subsection{ADS-B Checksum}\label{ads-b-checksum}

ADS-B uses a cyclic redundancy check to validate the correctness of the received message, where the last 24 bits are the parity bits. The following pseudo-code describes the CRC process:

\begin{verbatim}
GENERATOR = 1111111111111010000001001

MSG = binary("8D4840D6202CC371C32CE0576098")  # total 112 bits

FOR i FROM 0 TO 88:                           # 112 - 24 parity bits
  if MSG[i] is 1:
    MSG[i:i+24] = MSG[i:i+24] ^ GENERATOR

CRC = MSG[-24:]                               # last 24 bits

IF CRC not 0:
  MSG is corrupted

\end{verbatim}

For the implementation of CRC encoder in Python, refer to the pyModeS library function: \texttt{pyModeS.crc()}

A comprehensive documentation on Mode-S parity coding can be found:

\begin{verbatim}
Gertz, Jeffrey L. Fundamentals of mode s parity coding. No. ATC-117.
MASSACHUSETTS INST OF TECH LEXINGTON LINCOLN LAB, 1984. APA
\end{verbatim}
