\chapter{Summary and beyond}

\section{Summary}
Finally, we have come to the end of this book. We started the book by introducing the most fundamental knowledge of modern radar technologies, establishing the technical background of Mode~S and ADS-B signals, and describing the hardware and software for receiving these signals. By the end of Part 1, we are able to set up a basic system, including the antenna, software-defined radio, and software tool. One thing worth noting is that I only illustrated the process of setting up two open-source tools under a Linux operation system. But they can certainly be used under other different operating systems.

The main focus of the book (Part 2 and 3) is on explaining the detailed structures of all ADS-B and common Mode~S downlink messages, as well as illustrating the decoding process with step-by-step examples when necessary. I tried to cover as many fields as possible in these messages. However, there might be certain fields that lack detailed explanation. The ICAO Annex IV and ICAO Doc 9871 are always two good sources to find out more information.

In Part 2, we first introduced the basics of ADS-B, such as message structure, types, and different ADS-B versions. We then explained the structures and decoding processes for all ADS-B messages, including identification and category, airborne position, surface position, airborne velocity, and aircraft operation status. We also talked about uncertainties parameters like uncertainties, accuracies, and integrities of ADS-B positions and velocities. Finally, in Part 2, we described the error control mechanism of ADS-B messages, which can be used to detect (and potentially corrected) corrupt messages.

In Part 3, we focused on the decoding of other types of Mode~S messages. After explaining the basics of Mode~S, we discussed all-call reply, surveillance replies, ACAS messages, and Comm-B messages. These are messages transmitted in different Mode~S downlink format. Comm-B downlink is the one that contains the most information. In theory, it supports up to 256 messages types (not all are used currently). Within Comm-B communication, some subsets of messages are commonly used by air traffic controllers. Thus, we continue with the decoding of nine message types in three groups of Mode~S services, which are Mode~S elementary surveillance, Mode~S enhanced surveillance, and Mode~S meteorological services. As the third party receiver, the difficulty of decoding Comm-B messages is that we can not easily identify the types of messages like ADS-B messages. So, in the last chapter of this part, we discussed the important aspect of how to infer Comm-B types from downlink messages.


\section{Crowd-sourced networks}

In chapter 2, we talked about the coverage of ADS-B receivers. Even with a perfect location, one receiver can cover a maximum area of around 400 radius, due to the curvature of the Earth. When the antenna is not free of obstacles like buildings and mountains, the coverage is further reduced. 

To cover full flight trajectories for a larger area. A group of receivers needs to be deployed. Furthermore, in order to have coverage over the entire continent (or multiple continents), an even larger network of receivers is needed. To deploy and maintain receivers on such a larger scale is difficult for any individual entity. That is the reason for crowd-sourced ADS-B networks. 

A crowd-sourced network usually consists of a large group of enthusiasts who are voluntarily feeding their receiver data to a central server (or cluster of servers) on the internet owned by an entity. The owner of the service can be commercial companies or non-profit organizations. These companies often provide free premium user accounts to the contributors in return for their data. Some companies also provide free receivers for users to host in areas with less coverage. Commonly, in this case, they make agreements with receiver hosts to maintain and keep the receiver online as much as possible. In Table \ref{tb:adsb_networks}, several ADS-B networks are listed.

\begin{table}[ht]
\caption{A short list of ADS-B receiver networks}
\label{tb:adsb_networks}
\begin{threeparttable}
\begin{tabular}{|l|l|l|l|}
\hline
\textbf{Name} & \textbf{Type} & \textbf{Founded} & \textbf{Number of receivers} \\ \hline
FlightAware & Commercial & 2005 & 30,000\tnote{*} \\ \hline
FlightRadar24 & Commercial & 2006 & 25,000\tnote{*} \\ \hline
RadarBox24 & Commercial & 2007 & unknown \\ \hline
OpenSky network & Non-profit / research & 2013 & 3,400\tnote{*} \\ \hline
ADS-B Exchange & Non-profit & 2016 & unknown \\ \hline
\end{tabular}
\begin{tablenotes}
\item[*] Number of receivers obtained from each company's website, as of October 2020.
\end{tablenotes}
\end{threeparttable}    
\end{table}

Another advantage of having a network of receivers, it is also possible to locate aircraft or validate the location of aircraft using any signals transmitted by aircraft with multilateration\cite{kaune2012}. This technique has been used by many receiver networks. It is specifically useful to locate aircraft that are not equipped with ADS-B compliant transponders.

\subsection{OpenSky network}

Among these crowd-sourced networks, the OpenSky network \footnote{https://opensky-network.org} is one of the most well-known research networks in the air transportation research community. It is a crowd-sourced network that is fully formed by volunteers and researchers around the world. Currently, it has quite good coverage over Europe and a large part of North America. In Figure \ref{fig:opensky_coverage}, we can see the current global coverage of the OpenSky network.

\begin{figure}[ht]
    \centering
    \includegraphics[width=\columnwidth]{figures/conclusion/opensky.png}
    \caption{Coverage of OpenSky network, as of November 2020}
    \label{fig:opensky_coverage}
\end{figure}

The OpenSky network provides uses free and unlimited access to the data for non-commercial usages \cite{schafer2014opensky}. It is possible to access its live ADS-B traffic data through web APIs publicly. Access to historical data is also possible through its Impala shell. However, the user must request this permission first. I also wrote a Python library,\footnote{https://github.com/junzis/pyopensky} which facilitates the access and decoding of raw Mode~S messages from OpenSky historical database using its Impala interface \cite{sun2019pyopensky}. 

If you are setting up a receiver or have an existing receiver, I strongly encourage you considering sharing your Mode~S feed with the OpenSky network to promote open-access data from air transportation research.

\section{Additional data}

For many air traffic-related studies, we frequently have to rely on more information that is not transmitted in ADS-B and Mode~S. One of the first things we often need is an aircraft database that can map the transponder address from ADS-B or Mode~S to aircraft registration and aircraft type. 

A few countries, including the United States, make this information publicly available for download. However, a large part of this information is not officially available or behind paywalls. There are many different sources of aircraft data that are maintained by hobbyists and volunteers on the internet. A select list of data sources can be found on the \emph{Open Aviation Data} website.\footnote{https://atmdata.github.io/sources/}.

In this website, a few other common data sources are also listed, including, for example, aircraft performance database, engine emission data, and weather data. All these data sources are available for public use.


\section{Congestion}
In the first chapter of this book, we discussed all types of signals that are transmitted over the 1090 megahertz frequency. They include Mode~A, Mode~C, and Mode~S. Due to the mix of different types of transponders from a wide range of aircraft including both commercial and general aviation, all these modes of surveillance are still in use nowadays. In addition to the increasing number of operational aircraft, the frequency congestion can be a quite serious issue for surveillance data link during the day time in busy airspace. 

In a recent study \cite{sun2020rf}, we found that more than 60 \% of the ADS-B signals we received during the daytime are corrupted, within the airspace we are located. Figure \ref{fig:adsb_corruption} illustrate the severity of the frequency congestion during one day of radio frequency testing for the Dutch airspace.

\begin{figure}[ht]
    \centering
    \includegraphics[width=0.7\columnwidth]{figures/conclusion/adsb_corruption.pdf}
    \caption{Corruption detected in ADS-B messages during a 24 hour period on 29
    January, 2020.}
    \label{fig:adsb_corruption}
\end{figure}


\section{Future of ADS-B}
Reducing radio frequency congestion is one of the important areas for future research. Some potential solutions can help us mitigate this problem in the future, for example:

\begin{itemize}
    \item Reduce and phase out Mode~A and Mode~C surveillance.
    \item Balance Mode~S interrogations among different surveillance radars.
    \item Migrate interrogation based surveillance to broadcast (like ADS-B).
    \item Introduce more error resistance modulation and channel coding methods.
    \item Make use of alternative frequencies.
\end{itemize}

Another issue is the global coverage of ADS-B. Currently, it is still difficult to fully tracking aircraft over the oceans, due to the lack of ground station. One of the trends is to make use of satellites for receiving and relaying ADS-B messages transmitted by planes to ground stations \cite{noschese2011}. Companies like Aireon and spires are launching satellites that can receive ADS-B signals. 

It is still challenging for satellites to handle signals with a low signal to noise ratio due to the high altitude of the satellites' orbits, as well as high signal garbling rates due to their comprehensive coverage. 

Furthermore, given the restricted access to this large quantity of oceanic ADS-B data owned by a few companies behind paywalls,  it is hard for the research community to make use of this data. Whether they are willing to improve the openness of ADS-B ecosystem, is a question that remains to be answered.