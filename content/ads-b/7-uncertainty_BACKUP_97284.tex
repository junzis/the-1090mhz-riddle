\chapter{Uncertainties in ADS-B} \label{chap:uncertainty}

There are many parameters in ADS-B that define the data quality of the position and velocity reports. With each evolution of the ADS-B versions, these parameters have been renamed and redefined. This creates a complicated landscape to understand and to analyze these indicators. This chapter is designed to give a better overview and mapping among data quality parameters across different ADS-B versions.

\section{Terminology}
In general, there are three types of data quality indicators:

\begin{itemize}
  \item \textbf{Uncertainty indicators}: These indicators are introduced in version 0. Parameter values indicate that at least 95\% of measurements are within the allowed uncertainty bounds.
  \item \textbf{Accuracy indicators}: These indicators are first introduced in version 1 and are intended to replace previous uncertainty indicators. Parameter values also indicate that at least 95\% of measurements are within the allowed accuracy bounds.
  \item \textbf{Integrity indicators}: These indicators are also first introduced in version 1, replacing the uncertainty indicators from version 0.
\end{itemize}

Commonly, two parameters are related to the position and velocity separately in each group of indicators. Table \ref{tb:measurement-quality-indicators} shows the six major uncertainty indicators related ADS-B versions and their value ranges.

\begin{table}[ht]
\caption{Data quality indicators}
\label{tb:measurement-quality-indicators}
\begin{tabular}{|l|l|l|l|}
\hline
\textbf{Indicator} & \textbf{Acronym} & \textbf{Version} & \textbf{Values} \\ \hline
Navigation uncertainty category - position & NUCp & 0 & 0 - 9 \\ \hline
Navigation uncertainty category - rate (velocity) & NUCr & 0 & 0 - 4 \\ \hline
Navigation accuracy category - position & NACp & 1, 2 & 0 - 11 \\ \hline
Navigation accuracy category - velocity & NACv & 1, 2 & 0 - 4 \\ \hline
Navigation integrity category & NIC & 1, 2 & 0 - 11 \\ \hline
Surveillance integrity level & SIL & 1, 2 & 0 - 3 \\ \hline
\end{tabular}
\end{table}

Each quality indicator also relates to a set of parameters that express the exact amount of uncertainties, errors, or probabilities. In Table \ref{tb:uncertainty-parameters}, these parameters are listed.


\begin{table}[ht]
\caption{Parameters related with uncertainty, accuracy, and integrity}
\label{tb:uncertainty-parameters}
\begin{tabular}{|l|p{8cm}|l|}
\hline
\textbf{} & \textbf{Parameter} &  \\ \hline
NUCp (V0) & Horizontal protection limit & HPL \\ \cline{2-3}
 & 95\% containment radius on horizontal position error & RCxy \\ \cline{2-3}
 & 95\% containment radius on vertical position error & RCz \\ \hline
NUCr (V0) & Horizontal velocity error (95\%) & HVE \\ \cline{2-3}
 & Vertical velocity error (95\%) & VVE \\ \hline
NACp (V1,2) & Estimated position uncertainty (95\% horizontal accuracy - p). a.k.a. horizontal figure of merit (HFOM) in GNSS & EPU \\ \cline{2-3}
 & Vertical estimated position uncertainty (95\% vertical accuracy - p). a.k.a. vertical figure of merit (VFOM) in GNSS & VEPU \\ \hline
NACv (V1,2) & Horizontal figure of merit (95\% horizontal accuracy - v) & HFOMr \\ \cline{2-3}
 & Vertical figure of merit (95\% horizontal accuracy - v) & VFOMr \\ \hline
NIC (V1,2) & Horizontal containment radius limit & RC \\ \cline{2-3}
 & Vertical protection limit & VPL \\ \hline
SIL (V1,2) & Probability of exceeding horizontal containment radius & P-RC \\ \cline{2-3}
 & Probability of exceeding vertical integrity containment region & P-VPL \\ \hline
\end{tabular}
\end{table}

These parameters are identified by different bits from different messages. Due to the evolution of ADS-B versions, the definitions can also differ. In order to correctly obtain these parameters, the version of ADS-B transponder used must be identified. This can be based on the logic explained in chapter \ref{chap:adsb-basic}.

In the rest of this chapter, these parameters are explained in detail according to different ADS-B versions. The changes in the same parameters between different versions are also indicated.


\section{Version 0}

In ADS-B version 0, only uncertainties are defined. Two sets of parameters are related to the position and velocity (or rate) separately.

\subsection{Navigation uncertainty category - position (NUCp)}

NUCp is directly related to ADS-B Type Code with one-to-one mapping. The mapping between Type Code and NUCp in surface position messages and the two types of airborne position messages is shown in Table \ref{tb:tc-nucp-mapping}.


\begin{table}[ht]
\caption{Mapping between TC and NUCp}
\label{tb:tc-nucp-mapping}

\begin{subtable}[t]{0.7\linewidth}
\begin{tabular}{|l||l|l|l|l|l|}
\hline
 & \multicolumn{5}{l|}{Surface position}  \\ \hline
TC & 0 & 5 & 6 & 7 & 8 \\ \hline
NUCp & 0 & 9 & 8 & 7 & 6 \\ \hline
\end{tabular}
\end{subtable}

\vspace{0.2cm}

\begin{subtable}[t]{0.7\linewidth}
\begin{tabular}{|l||l|l|l|l|l|l|l|l|l|l|}
\hline
  & \multicolumn{10}{l|}{Airborne position (Barometric altitude)} \\ \hline
TC & 9 & 10 & 11 & 12 & 13 & 14 & 15 & 16 & 17 & 18 \\ \hline
NUCp & 9 & 8 & 7 & 6 & 5 & 4 & 3 & 2 & 1 & 0 \\ \hline
\end{tabular}
\end{subtable}

\vspace{0.2cm}

\begin{subtable}[t]{0.7\linewidth}
\begin{tabular}{|l||l|l|l|}
\hline
  & \multicolumn{3}{l|}{(GNSS altitude)} \\ \hline
TC & 20 & 21 & 22 \\ \hline
NUCp & 9 & 8 & 0 \\ \hline
\end{tabular}
\end{subtable}

\end{table}

In general, a higher NUCp number (lower TC number) represents higher confidence in the position measurement. When dealing with position uncertainty, the horizontal protection limit (HPL), the containment radius on horizontal position error (RCxy), and the containment radius on vertical position error (RCz) are used to quantify the uncertainties. All values are shown in Table \ref{tb:nucp-params}.

\begin{table}[!ht]
\caption{NUCp parameters and their values}
\label{tb:nucp-params}
\begin{tabular}{|l|l|l|l|l|}
\hline
\textbf{TC} & \textbf{NUCp} & \textbf{HPL} & \textbf{RCxy} & \textbf{RCz} \\ \hline \hline
0 & 0 & N/A & N/A & N/A \\ \hline
5 & 9 & \textless 7.5 m & \textless 3 m & N/A \\ \hline
6 & 8 & \textless 25 m & \textless 10 m & N/A \\ \hline
7 & 7 & \textless 0.1 NM (185 m) & \textless 0.05 NM (93 m) & N/A \\ \hline
8 & 6 & \textgreater 0.1 NM (185 m) & \textgreater 0.05 NM (93 m) & N/A \\ \hline
\hline
9 & 9 & \textless 7.5 m & \textless 3 m & N/A \\ \hline
10 & 8 & \textless 25 m & \textless 10 m & N/A \\ \hline
11 & 7 & \textless 0.1 NM (185 m) & \textless 0.05 NM (93 m) & N/A \\ \hline
12 & 6 & \textless 0.2 NM (370 m) & \textless 0.1 NM (185 m) & N/A \\ \hline
13 & 5 & \textless 0.5 NM (926 m) & \textless 0.25 NM (463 m) & N/A \\ \hline
14 & 4 & \textless 1 NM (1852 m) & \textless 0.5 NM (926 m) & N/A \\ \hline
15 & 3 & \textless 2 NM (3704 m) & \textless 1 NM (1852 m) & N/A \\ \hline
16 & 2 & \textless 10 NM (18520 m) & \textless 5 NM (9260 m) & N/A \\ \hline
17 & 1 & \textless 20 NM (37040 m) & \textless 10 NM (18520 m) & N/A \\ \hline
18 & 0 & \textgreater 20 NM (37040 m) & \textgreater 10 NM (18520 m) & N/A \\ \hline
\hline
20 & 9 & \textless 7.5 m & \textless 3 m & \textless 4 m \\ \hline
21 & 8 & \textless 25 m & \textless 10 m & \textless 15 m \\ \hline
22 & 0 & \textgreater 25 m & \textgreater 10 m & \textgreater 15 m \\ \hline
\end{tabular}
\end{table}

It is worth noting that, in the case of airborne position with GNSS height (TC=20-22), HPL and RCxy are defined with slight differences compared to other types of position messages. In addition, the containment radius for the vertical position (\texttt{RCv}) is also defined in this message. This is possible because the altitude is obtained from GNSS sources.


\subsection{Navigation uncertainty category - rate (NUCr)}

The Navigation Uncertainty Category - rate (NUCr) is used to indicate the uncertainty of the horizontal and vertical speeds. The bits representing NUCr can be found in the airborne velocity message (TC=19). The NUCr is located at message bit 43-45 (or ME bit 11-13), which defines the 95\% of the error in horizontal and vertical speed as shown in Table \ref{tb:nucr-params}.

\begin{table}[ht]
\caption{NUCr parameters and their values}
\label{tb:nucr-params}
\begin{tabular}{|l|l|l|}
\hline
\textbf{NUCp} & \textbf{HVE (95\%)} & \textbf{VVE (95\%)} \\ \hline
0 & N/A & N/A \\ \hline
1 & \textless 10 m/s & \textless 15.2 m/s (50 fps) \\ \hline
2 & \textless 3 m/s & \textless 4.5 m/s (15 fps) \\ \hline
3 & \textless 1 m/s & \textless 1.5 m/s (5 fps) \\ \hline
4 & \textless 0.3 m/s & \textless 0.46 m/s (1.5 fps) \\ \hline
\end{tabular}
\end{table}



\section{Version 1}


In ADS-B version 1, the uncertainty category is removed, replaced by the accuracy category and the integrity category. Both NUCp and NUCr from version 0 do not exist in version 1. New indicators introduced in version 1 are NACp, NACv, NIC, and SIL.


\subsection{Navigation integrity category (NIC)}

NIC is designed to replace NUCp from version 0, but with more levels included. Like NUCp in version 0, the Navigation Integrity Category (NIC) is related to the Type Code. However, Type Code and NIC do not have a one-to-one mapping relationship. With more levels defined, a supplemental bit is required to distinguish two levels represented by some of the same Type Codes.

The NIC Supplement bit (NICs) is introduced in the operation status messages (TC=31) and located at the message bit 76 (or ME bit 44). The relationship between TC, NIC, and Rc are listed in Table \ref{tb:nic-params-v1}.

\begin{table}[!ht]
\caption{NIC parameters and their values (version 1)}
\label{tb:nic-params-v1}
\begin{tabular}{|l|l|l|l|l|}
\hline
\textbf{TC} & \textbf{NICs} & \textbf{NIC} & \textbf{RC} & \textbf{VPL} \\ \hline \hline
0 & N/A & N/A & N/A & N/A \\ \hline
5 & 0 & 11 & \textless 7.5 m & N/A \\ \hline
6 & 0 & 10 & \textless 25 m & N/A \\ \hline
7 & 1 & 9 & \textless 75 m & N/A \\ \cline{2-5}
 & 0 & 8 & \textless 0.1 NM (185 m) & N/A \\ \hline
8 & 0 & 0 & \textgreater 0.1 NM or Unknown & N/A \\ \hline
\hline
9 & 0 & 11 & \textless 7.5 m & \textless 11 m \\ \hline
10 & 0 & 10 & \textless 25 m & \textless 37.5 m \\ \hline
11 & 1 & 9 & \textless 75 m & \textless 112 m \\ \cline{2-5}
 & 0 & 8 & \textless 0.1 NM (185 m) & N/A \\ \hline
12 & 0 & 7 & \textless 0.2 NM (370 m) & N/A \\ \hline
13 & 0 & 6 & \textless 0.5 NM (926 m) & N/A \\ \cline{2-2} \cline{4-5}
 & 1 &  & \textless 0.6 NM (1111 m) & N/A \\ \hline
14 & 0 & 5 & \textless 1.0 NM (1852 m) & N/A \\ \hline
15 & 0 & 4 & \textless 2 NM (3704 m) & N/A \\ \hline
16 & 1 & 3 & \textless 4 NM (7408 m) & N/A \\ \cline{2-5}
 & 0 & 2 & \textless 8 NM (14.8 km) & N/A \\ \hline
17 & 0 & 1 & \textless 20 NM (37.0 km) & N/A \\ \hline
18 & 0 & 0 & \textgreater 20 NM or Unknown & N/A \\ \hline
\hline
20 & 0 & 11 & \textless 7.5 m & \textless 11 m \\ \hline
21 & 0 & 10 & \textless 25 m & \textless 37.5 m \\ \hline
22 & 0 & 0 & \textgreater 25 m & \textgreater 112 m \\ \hline
\end{tabular}
\end{table}

\subsection{Navigation accuracy category - position (NACp)}

NACp is introduced in ADS-B version 1 as a complementary indicator of NIC. The NACp can be obtained from the operational status message, message frame bits 77-80 (or ME bits 45-48).

With NACp, the 95\% horizontal and vertical accuracy bounds can be determined. They are the estimated position uncertainty (EPU) and the vertical estimated position uncertainty (VEPU), which are also known as the horizontal figure of merit (HFOM) and the vertical figure of merit (VFOM) in GNSS. 

EPU and Rc in NIC have the following relationship:

\begin{equation}
  \begin{split}
    \mathrm{EPU} &\approx \mathrm{Rc} / 2.5   \qquad  \text{for NACp} \ge 9 \\
    \mathrm{EPU} &\approx \mathrm{Rc} / 2.0  \qquad  \text{for NACp} < 9
  \end{split}
\end{equation}

NACp and its related parameter values are defined in Table \ref{tb:nacp-params}.

\begin{table}[]
\caption{NACp parameters and their values}
\label{tb:nacp-params}
\begin{tabular}{|l|l|l|}
\hline
\textbf{NACp} & \textbf{EPU (or HFOM)} & \textbf{VEPU (or VFOM)} \\ \hline \hline
11 & \textless 3 m & \textless 4 m \\ \hline
10 & \textless 10 m & \textless 15 m \\ \hline
9 & \textless 30 m & \textless 45 m \\ \hline
8 & \textless 0.05 NM (93 m) & N/A \\ \hline
7 & \textless 0.1 NM (185 m) & N/A \\ \hline
6 & \textless 0.3 NM (556 m) & N/A \\ \hline
5 & \textless 0.5 NM (926 m) & N/A \\ \hline
4 & \textless 1.0 NM (1852 m) & N/A \\ \hline
3 & \textless 2 NM (3704 m) & N/A \\ \hline
2 & \textless 4 NM (7408 m) & N/A \\ \hline
1 & \textless 10 NM (18520 m) & N/A \\ \hline
0 & \textgreater 10 NM or Unknown & N/A \\ \hline
\end{tabular}
\end{table}



\subsection{Navigation accuracy category - velocity (NACv)}

NACv is introduced in version 1 to replace the NUCv from version 0. The bits are located at the same location and have the same definitions of values. These bits are contained in airborne velocity message (TC=19) message bits 43-45 (or ME bits 11-13). It defines the 95\% of the errors in horizontal and vertical speeds. The detailed definitions of the Horizontal Figure of Merit for rate (HFOMr) and Vertical Figure of Merit for rate (HFOMr) are shown in Table \ref{tb:nacv-params}.

\begin{table}[!ht]
\caption{NACv parameters and their values}
\label{tb:nacv-params}
\begin{tabular}{|l|l|l|}
\hline
\textbf{NACv} & \textbf{HFOMr} & \textbf{VFOMr} \\ \hline
0 & N/A & N/A \\ \hline
1 & \textless 10 m/s & \textless 15.2 m/s (50 fps) \\ \hline
2 & \textless 3 m/s & \textless 4.5 m/s (15 fps) \\ \hline
3 & \textless 1 m/s & \textless 1.5 m/s (5 fps) \\ \hline
4 & \textless 0.3 m/s & \textless 0.46 m/s (1.5 fps) \\ \hline
\end{tabular}
\end{table}


\subsection{Surveillance integrity level (SIL)}

SIL is introduced in version 1 and used to indicate the probability of measurements exceeding the containment radius. The SIL value can also be found in the operational status message (TC=31), message bits 83-84 (or ME bits 51-52)

Each SIL value corresponds to two probabilities that describe the horizontal (P-RC) and vertical (P-VPL) components respectively. The definitions are as follows:

\begin{table}[!ht]
\caption{SIL parameters and their values}
\label{tb:sil-params}
\begin{tabular}{|l|l|l|}
\hline
SIL & P-RC & P-VPL \\ \hline
0 & unknown & unknown \\ \hline
1 & $< 1 \times 10^{-3}$ & $< 1 \times 10^{-3}$ \\ \hline
2 & $< 1 \times 10^{-5}$ & $< 1 \times 10^{-5}$ \\ \hline
3 & $< 1 \times 10^{-7}$ & $< 2 \times 10^{-7}$ \\ \hline
\end{tabular}
\end{table}

The unit for P-RCu and P-VPL can be per flight hour or per
sample, except when SIL=3, the unit for P-VPL becomes per 150 seconds per sample.



\section{Version 2}

There are fewer changes between version 1 and version 2 when compared to the earlier update. The major changes are the further refined NIC levels and minor updates of SIL parameters.

\subsection{Navigation integrity category (NIC)}

In version 2, NIC levels can be obtained with three additional supplement bits named NIC supplement bit A (NICa), NIC supplement bit B (NICb), and NIC supplement bit C (NICc). These bits can be found as follows:


\begin{itemize}
  \item NICa is located in the operational status message (TC=31) (message bit 76 or ME bit 44), which is the same as in ADS-B version 1.
  \item NICb is located in the airborne position message (TC=9-18), message bit 40 or ME bit 8), where the Single Antenna Flag was located in previous ADS-B versions.
  \item NICc is also located in the operational status message (TC=31) (message bit 52 or ME bit 20).
\end{itemize}

With these supplemental bits and the Type Code, the NIC value in version 2 can be calculated. NIC values and their related Rc values are shown in Table \ref{tb:nic-params-v2}.


\begin{table}[!ht]
\caption{NIC parameters and their values (version 2)}
\label{tb:nic-params-v2}
\begin{tabular}{|l|l|l|l|l|l|}
\hline
\textbf{TC} & \textbf{NICa} & \textbf{NICb} & \textbf{NICc} & \textbf{NIC} & \textbf{Rc} \\ \hline \hline
5 & 0 & N/A & 0 & 11 & \textless 7.5 m \\ \hline
6 & 0 & N/A & 0 & 10 & \textless 25 m \\ \hline
7 & 1 & N/A & 0 & 9 & \textless 75 m \\ \cline{2-6}
 & 0 & N/A & 0 & 8 & \textless 0.1 NM (185 m) \\ \hline
8 & 1 & N/A & 1 & 7 & \textless 0.2 NM (370 m) \\ \cline{2-6}
 & 1 & N/A & 0 & 6 & \textless 0.3 NM (556 m) \\ \cline{2-4} \cline{6-6}
 & 0 & N/A & 1 &  & \textless 0.6 NM (1111 m) \\ \cline{2-6}
 & 0 & N/A & 0 & 0 & \textgreater 0.6 NM or unknown \\ \hline
9 & 0 & 0 & N/A & 11 & \textless 7.5 m \\ \hline
10 & 0 & 0 & N/A & 10 & \textless 25 m \\ \hline
11 & 1 & 1 & N/A & 9 & \textless 75 m \\ \cline{2-6}
 & 0 & 0 & N/A & 8 & \textless 0.1 NM (185 m) \\ \hline
12 & 0 & 0 & N/A & 7 & \textless 0.2 NM (370 m) \\ \hline
13 & 0 & 1 & N/A & 6 & \textless 0.3 NM (556 m) \\ \cline{2-4} \cline{6-6}
 & 0 & 0 & N/A &  & \textless 0.5 NM (926 m) \\ \cline{2-4} \cline{6-6}
 & 1 & 1 & N/A &  & \textless 0.6 NM (1111 m) \\ \hline
14 & 0 & 0 & N/A & 5 & \textless 1.0 NM (1852 m) \\ \hline
15 & 0 & 0 & N/A & 4 & \textless 2 NM (3704 m) \\ \hline
16 & 1 & 1 & N/A & 3 & \textless 4 NM (7408 m) \\ \cline{2-6}
 & 0 & 0 & N/A & 2 & \textless 8 NM (14.8 km) \\ \hline
17 & 0 & 0 & N/A & 1 & \textless 20 NM (37.0 km) \\ \hline
18 & 0 & 0 & N/A & 0 & \textgreater 20 NM or unknown \\ \hline
20 & N/A & N/A & N/A & 11 & \textless 7.5 m \\ \hline
21 & N/A & N/A & N/A & 10 & \textless 25 m \\ \hline
22 & N/A & N/A & N/A & 0 & \textgreater 25 m \\ \hline
\end{tabular}
\end{table}


\subsection{Surveillance integrity level (SIL)}

In version 2, an additional SIL supplement bit (SILs) is introduced to better distinguish whether the value has the unit of per flight hour or per sample. The SILs bit can also be found in the operational status message, message bit 87 (or ME bit 55). The definitions are:

\begin{itemize}
  \item SILs=0: probability is \emph{per hour} based
  \item SILs=1: probability is \emph{per sample} based
\end{itemize}


The values related to SIL remain the same as what is shown in Table \ref{tb:sil-params}.

\subsection{NACp and NACv}

NACp and NACv in version 2 remain the same as in version 1. Related parameters and definitions can be found in the previous Table \ref{tb:nacp-params} and Table \ref{tb:nacv-params} respectively.
