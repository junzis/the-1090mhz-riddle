\chapter{Aircraft Operation Status}\label{aircraft-operation-status}

The aircraft operational status message is designed to provide various information on an aircraft. The message has been defined in all ADS-B versions. But in practice, it is not implemented in ADS-B version 0. In ADS-B version 0, the structures include information related to aircraft operational capabilities and status for different stages of the flight. From version 1 onward, the operational status includes more information, such as ADS-B version, accuracy, and integrity indicators.

The operation status message is transmitted with Type Code 31 (\texttt{TC=31}). The structures of this message type differ significantly over different ADS-B versions.

\section{Version 0}

In version 0, the structure of the message is shown in Table \ref{tb:adsb-operational-status-v0}. Two main blocks contain the status information, which includes operational capabilities (16 bits) and operational status (16 bits). Each block contains 4 parameters, and each parameter is encoded with 4 bits.

\begin{table}[ht]
\caption{Aircraft operational status (Version 0)}
\label{tb:adsb-operational-status-v0}
\footnotesize
\begin{tabular}{|l|c|c|c|c|}
\hline
\textbf{FIELD} & \textbf{ABBR} & \textbf{FB} & \textbf{MB} & \textbf{N-BITS} \\ \hline
\begin{tabular}[c]{@{}l@{}}Type code = 31 \\ Binary: 11111\end{tabular} & TC & \begin{tabular}[c]{@{}c@{}}33\\ :\\ 37\end{tabular} & \begin{tabular}[c]{@{}c@{}}1\\ :\\ 5\end{tabular} & 5 \\ \hline
\begin{tabular}[c]{@{}l@{}}Sub-type code = 0\\ Binary: 000\end{tabular} & ST & \begin{tabular}[c]{@{}c@{}}38\\ :\\ 40\end{tabular} & \begin{tabular}[c]{@{}c@{}}6\\ :\\ 8\end{tabular} & 3 \\ \hline
\begin{tabular}[c]{@{}l@{}}Enroute operational capabilities (4 bits)\\ Terminal area operational capabilities (4 bits)\\ Approach/landing operational capabilities (4 bits)\\ Surface operational capabilities (4 bits)\end{tabular} & \begin{tabular}[c]{@{}c@{}}CC4\\ CC3\\ CC2\\ CC1\end{tabular} & \begin{tabular}[c]{@{}c@{}}41\\ :\\ 56\end{tabular} & \begin{tabular}[c]{@{}c@{}}9\\ :\\ 24\end{tabular} & 16 \\ \hline
\begin{tabular}[c]{@{}l@{}}Enroute operational status (4 bits)\\ Terminal area operational status (4 bits)\\ Approach/landing operational status (4 bits)\\ Surface operational status (4 bits)\end{tabular} & \begin{tabular}[c]{@{}c@{}}OM4\\ OM3\\ OM2\\ OM1\end{tabular} & \begin{tabular}[c]{@{}c@{}}57\\ :\\ 72\end{tabular} & \begin{tabular}[c]{@{}c@{}}25\\ :\\ 40\end{tabular} & 16 \\ \hline
Reserved &  & \begin{tabular}[c]{@{}c@{}}73\\ :\\ 88\end{tabular} & \begin{tabular}[c]{@{}c@{}}41\\ :\\ 56\end{tabular} & 16 \\ \hline
\end{tabular}
\end{table}

However, even all the parameter fields are defined. All 
\texttt{CC} and \text{OM} bits are reserved in the third and fourth blocks according to \cite{icao9871v1}, which eventually made the definitions un-usable for version 0 transponders. As version 0 transponders do not transmit any operation status messages, the absence of TC=31 messages can help identify transponder version 0, which is discussed in chapter \ref{chap:adsb-basic}.

\section{Version 1}

From version 1 onward, the operational status report is implemented and broadcast by the transponder. In Table \ref{tb:adsb-operational-status-v1}, the structure of operational message in version 1 is defined.

\begin{table}[ht]
\caption{Aircraft operational status (Version 1)}
\label{tb:adsb-operational-status-v1}
\footnotesize
\begin{tabular}{|l|c|c|c|c|}
\hline
\textbf{FIELD} & \textbf{ABBR} & \textbf{FB} & \textbf{MB} & \textbf{N-BITS} \\ \hline \hline
\begin{tabular}[c]{@{}l@{}}\textbf{Type code = 31} \\ Binary: 11111\end{tabular} & TC & \begin{tabular}[c]{@{}c@{}}33\\ :\\ 37\end{tabular} & \begin{tabular}[c]{@{}c@{}}1\\ :\\ 5\end{tabular} & 5 \\ \hline
\begin{tabular}[c]{@{}l@{}}\textbf{Sub-type code}\\ 0: airborne\\ 1: surface\\ 2-7: reserved\end{tabular} & ST & \begin{tabular}[c]{@{}c@{}}38\\ :\\ 40\end{tabular} & \begin{tabular}[c]{@{}c@{}}6\\ :\\ 8\end{tabular} & 3 \\ \hline
\begin{tabular}[c]{@{}l@{}}\textbf{ST=0: Airborne capacity class codes} (16 bits)\\ \textbf{ST=1: Surface capacity codes} (12 bits) \\ \quad\qquad+ \textbf{Length/width code} (4 bits)\end{tabular} & CC & \begin{tabular}[c]{@{}c@{}}41\\ :\\ 56\end{tabular} & \begin{tabular}[c]{@{}c@{}}9\\ :\\ 24\end{tabular} & 16 \\ \hline
\textbf{Operational mode codes} & OM & \begin{tabular}[c]{@{}c@{}}57\\ :\\ 72\end{tabular} & \begin{tabular}[c]{@{}c@{}}25\\ :\\ 40\end{tabular} & 16 \\ \hline
\begin{tabular}[c]{@{}l@{}}\textbf{ADS-B version number}\\ 0: Conformant to DOC 9871, Appendix A\\ 1: Conformant to DOC 9871, Appendix B\\ 2-7: Reserved\end{tabular} & VER & \begin{tabular}[c]{@{}c@{}}73\\ :\\ 75\end{tabular} & \begin{tabular}[c]{@{}c@{}}41\\ :\\ 43\end{tabular} & 3 \\ \hline
\textbf{NIC supplement} & NICs & 76 & 44 & 1 \\ \hline
\textbf{Navigational accuracy category - position} & NACp & \begin{tabular}[c]{@{}c@{}}45\\ :\\ 48\end{tabular} & \begin{tabular}[c]{@{}c@{}}77\\ :\\ 80\end{tabular} & 4 \\ \hline
\begin{tabular}[c]{@{}l@{}}\textbf{ST=0: Barometric altitude quality}\\ \textbf{ST=1: Reserved}\end{tabular} & \begin{tabular}[c]{@{}c@{}}BAQ\\ --\end{tabular} & \begin{tabular}[c]{@{}c@{}}81\\ 82\end{tabular} & \begin{tabular}[c]{@{}c@{}}49\\ 50\end{tabular} & 2 \\ \hline
\textbf{Surveillance integrity level} & SIL & \begin{tabular}[c]{@{}c@{}}83\\ 84\end{tabular} & \begin{tabular}[c]{@{}c@{}}51\\ 52\end{tabular} & 2 \\ \hline
\begin{tabular}[c]{@{}l@{}}\textbf{ST=0: Barometric altitude integrity}\\ \textbf{ST=1: Track angle or heading}\end{tabular} & \begin{tabular}[c]{@{}c@{}}BICbaro\\ T/H\end{tabular} & 85 & 53 & 1 \\ \hline
\textbf{Horizontal reference direction} & HRD & 86 & 54 & 1 \\ \hline
\textbf{Reserved} &  & \begin{tabular}[c]{@{}c@{}}87\\ 88\end{tabular} & \begin{tabular}[c]{@{}c@{}}55\\ 56\end{tabular} & 2 \\ \hline
\end{tabular}
\end{table}

Bits 73 to 86, which are unused in the previous version, are now defined with new parameters. These parameters include ADS-B version number and various indicators related to the accuracy of the state measurements broadcast by the aircraft in other types of ADS-B messages.

It is worth noting that the definitions of some fields with the sub-type of the message, which differentiate between airborne status messages and surface status messages.


\section{Version 2}

Compared to the previous update, the changes in the operational status report from version 1 to 2 are minimal. Several fields are renamed and a SIL supplement bit is added to bit-87. The structure is shown in Table \ref{tb:adsb-operational-status-v2}.


\begin{table}[ht]
\caption{Aircraft operational status (Version 2)}
\label{tb:adsb-operational-status-v2}
\footnotesize
\begin{tabular}{|l|c|c|c|c|}
\hline
\textbf{FIELD} & \textbf{ABBR} & \textbf{FB} & \textbf{MB} & \textbf{N-BITS} \\ \hline
\begin{tabular}[c]{@{}l@{}}\textbf{Type code = 31} \\ Binary: 11111\end{tabular} & TC & \begin{tabular}[c]{@{}c@{}}33\\ :\\ 37\end{tabular} & \begin{tabular}[c]{@{}c@{}}1\\ :\\ 5\end{tabular} & 5 \\ \hline
\begin{tabular}[c]{@{}l@{}}\textbf{Sub-type code}\\ 0: Airborne status message\\ 1: Surface status message\end{tabular} & ST & \begin{tabular}[c]{@{}c@{}}38\\ :\\ 40\end{tabular} & \begin{tabular}[c]{@{}c@{}}6\\ :\\ 8\end{tabular} & 3 \\ \hline
\begin{tabular}[c]{@{}l@{}}\textbf{ST=0: Airborne capacity class codes} (16 bits)\\ \textbf{ST=1: Surface capacity codes} (12 bits) \\ \quad\qquad+ \textbf{Length/width code} (4 bits)\end{tabular} & CC & \begin{tabular}[c]{@{}c@{}}41\\ :\\ 56\end{tabular} & \begin{tabular}[c]{@{}c@{}}9\\ :\\ 24\end{tabular} & 16 \\ \hline
\begin{tabular}[c]{@{}l@{}}\textbf{ST=0: Airborne operational mode codes}\\ \textbf{ST=1: Surface operational mode codes}\end{tabular} & OM & \begin{tabular}[c]{@{}c@{}}57\\ :\\ 72\end{tabular} & \begin{tabular}[c]{@{}c@{}}25\\ :\\ 40\end{tabular} & 16 \\ \hline
\begin{tabular}[c]{@{}l@{}}\textbf{ADS-B version number}\\ 0: Conformance with DOC 9871, Appendix A\\ 1: Conformance with DOC 9871, Appendix B\\ 2: Conformance with DOC 9871, Appendix C\\ 3-7: Reserved\end{tabular} & VER & \begin{tabular}[c]{@{}c@{}}73\\ :\\ 75\end{tabular} & \begin{tabular}[c]{@{}c@{}}41\\ :\\ 43\end{tabular} & 3 \\ \hline
\textbf{NIC supplement - A} & NICa & 76 & 44 &  \\ \hline
\textbf{Navigational accuracy category - position} & NACp & \begin{tabular}[c]{@{}c@{}}45\\ :\\ 48\end{tabular} & \begin{tabular}[c]{@{}c@{}}77\\ :\\ 80\end{tabular} & 4 \\ \hline
\begin{tabular}[c]{@{}l@{}}\textbf{ST=0: Geometric vertical accuracy}\\ \textbf{ST=1: Reserved}\end{tabular} & \begin{tabular}[c]{@{}c@{}}GVA\\ --\end{tabular} & \begin{tabular}[c]{@{}c@{}}81\\ 82\end{tabular} & \begin{tabular}[c]{@{}c@{}}49\\ 50\end{tabular} & 2 \\ \hline
\textbf{Source integrity level} & SIL & \begin{tabular}[c]{@{}c@{}}83\\ 84\end{tabular} & \begin{tabular}[c]{@{}c@{}}51\\ 52\end{tabular} & 2 \\ \hline
\begin{tabular}[c]{@{}l@{}}\textbf{ST=0: Barometric altitude integrity} \\ \textbf{ST=1: Track angle or heading}\end{tabular} & \begin{tabular}[c]{@{}c@{}}NICbaro\\ T/H\end{tabular} & 85 & 53 & 1 \\ \hline
\textbf{Horizontal reference direction} & HRD & 86 & 54 & 1 \\ \hline
\textbf{SIL supplement} & SILs & 87 & 55 & 1 \\ \hline
\textbf{Reserved} &  & 88 & 56 & 1 \\ \hline
\end{tabular}
\end{table}
